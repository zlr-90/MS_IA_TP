\documentclass[a4paper,11pt]{article}

\usepackage{../../../../sty/examen}
\usepackage{../../../../sty/shortcuts_js}
\usepackage{enumitem}


\newcommand{\titrecours}{numéro 1}
\newcommand{\titretd}{MDI720}
\newcommand{\auteur}{François Portier}
\newcommand{\annee}{2020/2021}
\newcommand{\ecole}{TELECOM ParisTech}% \titre{}
\newcommand{\titre}{Devoir maison~}% \titre{}


\begin{document}
\sloppy
\feuille{1}

%Pour ce TP de test de la plate-forme ``Classgrade'' vous devez déposer un \textbf{\underline{unique}} fichier \textbf{\underline{anonymisé}} (votre nom ne doit apparaître nulle part y compris dans le nom du fichier lui-même) sous format \lstinline+ipynb+ sur le site \url{http://peergrade.enst.fr/}.

Vous devez charger votre fichier sur eCampus, avant le dimanche 18/10/2020 23h59.
%Entre le lundi 16/10/2017 et le dimanche 22/10/2017, 23h59, vous devrez noter trois copies qui vous seront assignées anonymement, en tenant compte du barème suivant pour chaque question:
%  \begin{itemize}
%    \item[$\bullet$] 0 (manquant/ non compris/ non fait/ insuffisant)
%    \item[$\bullet$] 1 (passable/partiellement satisfaisant)
%    \item[$\bullet$] 2 (bien)
%  \end{itemize}
%Ensuite, il faudra également remplir de la même manière les points de notation suivants:
%% La note totale est sur \textbf{20} points répartis comme suit:
%\begin{itemize}
%  \item[$\bullet$] aspect global de présentation: qualité de rédaction, d'orthographe, d'aspect de présentation, graphes, titres, etc. (Question 21).
%  \item[$\bullet$] aspect global du code: indentation, Style PEP8, lisibilité du code, commentaires adaptés (Question 22)
%  \item[$\bullet$] Point particulier: absence de bug sur votre machine (Question 23)
%\end{itemize}
%Des commentaires pourront être ajoutés question par question si vous en sentez le besoin ou l'utilité pour aider la personne notée \`a s'améliorer, et de manière obligatoire si vous ne mettez pas 2/2 \`a une question.
%Enfin, veillez \`a rester polis et courtois dans vos retours.

% Les personnes qui n'auront pas rentré leurs notes avant la limite obtiendront également \textbf{\underline{zéro}}.

% \noindent Retard: malus de \textbf{4} pts par tranche de 24h (sauf excuses validées par l'administration).
\begin{center}
\textbf{Rappel: aucun travail par mail accepté!}
\end{center}

\vspace{0.5cm}

\hrule

\vspace{0.3cm}

% %%%%%%%%%%%%%%%%%%%%%%%%%%%%%%%%%%%%%%%%%%%%%%%%%%%%%%%%%%%%%%%%%%%%%%%%%%%%%%%
% \section*{Rappels sur le modèle linéaire}
% %%%%%%%%%%%%%%%%%%%%%%%%%%%%%%%%%%%%%%%%%%%%%%%%%%%%%%%%%%%%%%%%%%%%%%%%%%%%%%%


% Nous considérons le modèle statistique suivant:
% \begin{equation}
% \label{eq:modeleLin}
% \bfy=X\bftheta+\bfepsilon,
% \end{equation}
% o\`{u}
% \begin{itemize}
% \item $\bfy=(y_i)_{1\leq i\leq n}\in\bbR^n$ est un vecteur colonne $n\times1$,
% \item $X=(X_{i,j})_{1\leq i\leq n,1\leq j\leq p}$ est une
%   matrice $n\times p$,
% \item $\bftheta=(\theta_i)_{1\leq i\leq p}\in\bbR^{p}$ est un
%   vecteur colonne $p\times1$,
% \item $\bfepsilon=(\varepsilon_i)_{1\leq i\leq n}\in\bbR^n$ un vecteur colonne $n\times1$ aléatoire.
% \end{itemize}
% % GAUSSIANITE?
% % Par simplicité, on suppose de plus que le vecteur $\bfepsilon$ suit la distribution :
% % $$
% % \bfepsilon\sim \cN(0,\sigma^2I_n)\ .
% % $$
% On rappelle les notations suivantes:
% \begin{itemize}\item
% $\hat{\bftheta} \in \argmin_{\theta \in \bbR^p} \|\bfy -X\theta\|^2/2$ l'estimateur par moindres carrés de $\bftheta$  (pour rappel $\hat{\bftheta} = (X^\top X)^{-1}X^\top \bfy$ quand la matrice $(X^\top X)$ est inversible.

% \item $\hat \bfy = X \hat\bftheta$, la prédiction sur les valeurs observées

% \item $\bar x_n = n^{-1}\sum_{i=1}^n x_i$ et $\bar y_n = n^{-1}\sum_{i=1}^n y_i$, les moyennes empiriques
% \item $\var_n(\bfx)$ et $\var_n(\bfy)$, les variances empiriques
% \item Le vecteur $r=\bfy-\hat \bfy$ est appelé vecteur des résidus.
% \item $\1_n=(1,\dots,1)^\top $ est le vecteur ``tout \`a un'' de taille $n\times 1$
% % \item On note
% %  $\textrm{SST} = || \bfy - \bar{y}_n\1_n||^2$, \quad% $\textrm{MST} = \textrm{SST}/(n-1)$
% %  $\textrm{SSR} = || \widehat \bfy - \bar{y}_n\1_n||^2$ (parfois aussi appelé SSM), \quad% $\textrm{MSM} =
% %  \textrm{SSM}/p$
%  % $\textrm{RSS} = || \bfy - \widehat \bfy||^2$ (\textit{Residual Sum of Squares} en anglais).

% % $\textrm{MSE} =  \textrm{SSE}/(n-p-1)$
% %\item On rappelle que SST = SSM  + SSE.


% % \item L'estimateur de la variance est $\hat\sigma^2 =\textrm{RSS}/(n-p-1)$.

% \end{itemize}

\input{../../../pool_exo/control_variates/control_variates}
%\input{../../../pool_exo/mco_galton/mco_galton_fr}
%\input{../../../pool_exo/auto-mpg/auto-mpg_fr}
% \input{../pool_exo/test_gaussien}
% \input{../pool_exo/IC_bootstrap}
% \input{../pool_exo/greedy_stepwise}
% \input{../pool_exo/greedy_math_corr}
% \input{../pool_exo/greedy_math}


% \bibliographystyle{apalike}
% \bibliography{../../../../sty/references_all}

\end{document}